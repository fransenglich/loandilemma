\documentclass[a4paper]{article}

\usepackage{dirtytalk}
\usepackage[english]{babel}
\usepackage[bookmarks=true]{hyperref}

\def\documenttitle{I Got Cash -- Shall I Down Pay My Student Loan?}

\title{\documenttitle}

\date{\today}

\author{Frans Englich \\
\href{mailto:fenglich@fastmail.fm}{fenglich@fastmail.fm}}

\hypersetup{
    pdfsubject = {\documenttitle},
    pdftitle = {\documenttitle}
}

\begin{document}

\maketitle

This paper discusses and explains the dilemma of \say{I've gotten a hold of a sum of
money. What is the wisest thing to do, shall I down pay my student loan all at once, or
shall I invest them instead?}

The answer is analyzed mathematically as well as discussed
informally.

The dilemma is that if you don't use your capital to pay down the loan at once,
you have capital over that you can invest, though you have to take into account
the running down payments of keeping the loan.

Hence, we discuss two different decisions:

$D_D$: Only down pay the whole loan at once, ignore any investment.

$D_I$: Invest in allegedly profitable investment $I$ while down paying the
payments of the loan.

\section{Mathematical Analysis}

First we define the student loan. It is a \emph{fixed annuity loan} with the
interest rate $L_r$, the fixed down payment amount $L_d$, and the withstanding
amount with present value $PV_w$. $PV_w$ is a cash flow, which is expressed as:

\def\PV_w{L_d \times \frac{1 - (r + 1)^{-n}}{r}}
\begin{equation}
PV_w=\PV_w
\end{equation}

This is contrasted to the investment, $I$. In order to compare it to the cost of
paying down the loan, we need to know its present value. Modelling or estimating
this amount is specific to the investment. If it's a stock, it can be estimated
in different ways. One approach to model the investment, is as the present net
value of cash flows:

\def\NPV_I{\sum_{t=0}^{N} \frac{B_t}{(1 + i)^t} - \sum_{t=0}^{N} \frac{C_t}{(1 + i)^t}}
\begin{equation}
NPV_I(i, N)= \NPV_I
\end{equation}

where $B_t$ are the benefits of the cashflow in time $t$, and $C_t$ is the cost
at time $t$. $N$ is the number of periods and $i$ is the discount rate. For
instance, the costs could be a one-time sum at time $0$, and the benefits could
be estimated dividends of a stock. We assume of course that the investment is
positive, that is $NPV_I(i, N) > 0$.

If the return of investment minus the down payments is positive, it is from an
exclusively financial perspective rational to do the investment; decision $D_I$.
Mathematically, it means this expression must be true:

\begin{equation}
    PV_w < NPV_I(i, N)
\end{equation}

When inserting the functions' bodies, it is:

\begin{equation}
    \PV_w < \NPV_I
\end{equation}

Here we see that there are many variables that affect this inequality:

$L_d$: the size of the down payment

$r$: The interest rate of the loan

$n$: The count of down payment

$B_t$: The benefits of the investment

$C_t$: The costs of the investment

$i$: The discount rate for the investment

$N$: The amount of cash flow periods.

In practice this inequality can be hard to approach and solve, because the
required variables can't be sufficiently estimated. For instance, the return(s)
of the investment is diffuse or that the interest rate of the loan may change
outside acceptable levels. One approach to mitigate this is a sensitivity
analysis.

\section{Additional Factors}

The rates need to be considered. Some may be variable and others fixed, and
hence inflation may become a factor. For instance, the loan may have a variable
rate and hence updating to inflation and monetary policy, while the investment
is fixed. The calculations need to take this into account.

Another aspect is liquidity. In this paper the investment, in decision $D_I$,
has an initial investment and subsequent returning cash flow(s). This, combined
with other relevant cash flows, most notably the down payment of the loan, means
there can be a liquidity problem. Stated differently, some period(s) can have a
negative cash flow. One way to investigate is to net all the cash flow periods.

But there's also psychological aspects, here the cognitive biases coming into
play. Having a loan involves risks, most notably not being able to pay the
down payments. The size of this risk can vary with various factors. Still,
mitigating this risk may be psychologically soothing, which can be justifiably
seen as a gain. However, it is prudent to weight it against any financial
benefit of deciding to do the investment.

Another aspect is the conditions of the loan. For instance student loans in
Norway and Sweden have advantageous aspects that needs to be taken into account.
This is for instance the possibility to postpone payments.

\section{Summary}

Answering the question of whether to down pay is combination between
mathematically consider the various options, combined with qualitative
questions.

\end{document}