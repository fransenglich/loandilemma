\documentclass[a4paper]{article}

\usepackage{dirtytalk}
\usepackage[english]{babel}
\usepackage[bookmarks=true]{hyperref}

\def\documenttitle{I Got Cash -- Shall I Down Pay My Student Loan?}

\title{\documenttitle}

\date{\today}

\author{Frans Englich \\
\href{mailto:fenglich@fastmail.fm}{fenglich@fastmail.fm}}

\hypersetup{
    pdfsubject = {\documenttitle},
    pdftitle = {\documenttitle}
}

\begin{document}

\maketitle

This paper discusses and explains the dilemma of \say{I've gotten a hold of a
sum of money. What is the wisest thing to do, shall I down pay my student loan
all at once, or shall I invest them instead?}

The dilemma is discussed qualitatively and analyzed mathematically. It is not an
exhaustive text, but looks from a handful of angles.

Hence, we discuss two different decisions:

$D_D$: Only down pay the whole loan at once, ignore any investment.

$D_I$: Invest in allegedly profitable investment $I$ while down paying the
payments of the loan.

\section{Qualitative Reasoning}

The dilemma is that if you don't use your capital to pay down the loan at once,
you have capital over that you can invest, though you have to take into account
the down payments of keeping the loan. By investing you can achieve a higher
overall return if that alternative is advantageous.

Interest rates and conditions of student loans are typically highly advantageous.

What may be confusing is that the debt -- your student loan -- stays the same
while what it funds differ if you decide to invest. If you invest, the loan
first funded your studies, while the same amount afterwards is used to fund the
investment.

If you keep the loan, the configuration can be seen as that the investment is
fully or partly funded by the loan which means you hence have a cost of capital
-- the interest rate -- which can be used in tools such as WACC. Notably the
return of the investment must be higher than the cost of capital -- the interest
rate of the loan -- for the investment to be profitable.

Also, whether the economy is subject to inflation or deflation is crucial. If
it's inflation, it means the money supply increases, but the loan does not,
which is advantageous. In contrast, if deflation resides, the loan's weight in
contrast to the money supply increases, and down paying would mitigate that.

Whether to invest must beyond the return also be considered from a perspective
of risk. In finance this is typically formulated using \emph{expected values},
meaning $E[r]$ would be considered, where $r$ is the return of the investment.

The advantage of down paying the loan is that it's the economically rational
thing to do if investing isn't profitable or feasible. It also comes with the
advantages associated with not having to down pay the loan periodically since
it's done at once.

\section{Mathematical Analysis}

First we define the student loan. It is a \emph{fixed annuity loan} with the
interest rate $L_r$, the fixed down payment amount $L_d$, and the withstanding
amount with present value $NPV_L$. $NPV_L$ is expressed as:


\begin{equation}
\label{eq:NPV_L}
NPV_L=L_d \frac{1 - (r + 1)^{-n}}{r}
\end{equation}

This is contrasted to the investment, $I$. In order to compare it to the cost
of paying down the loan, we need to know its present value. Modelling or
estimating this amount is specific to the investment, and that valuation can be
done in different ways. If it's for instance a stock, it can be estimated in
different ways. One approach to model the investment is as the present net
value of cash flows:


\begin{equation}
\label{eq:NPV_I}
NPV_I = {\sum_{t=0}^{N} \frac{B_t}{(1 + i)^t} -
         \sum_{t=0}^{N} \frac{C_t}{(1 + i)^t}}
\end{equation}

where $B_t$ are the benefits of the cash flow in time $t$, and $C_t$ is the cost
at time $t$. $N$ is the number of periods and $i$ is the discount rate. For
instance, the costs could be a one-time sum at time $0$, and the benefits could
be estimated dividends of a stock. We assume of course that the investment is
positive, that is $NPV_I(i, N) > 0$.

If the return of investment minus the down payments is positive, it is from an
exclusively financial perspective rational to do the investment; decision $D_I$.
Mathematically, it means this expression must be true:

\begin{equation}
0 < NPV_I - NPV_L
\end{equation}

We compare the present values and not the future values of these two
alternatives because the option is to down pay the loan today.

When inserting the expressions' bodies, it is:

\begin{equation}
\label{eq:NPV_ALL}
0 < {(\sum_{t=0}^{N} \frac{B_t}{(1 + r)^t} -
     \sum_{t=0}^{N} \frac{C_t}{(1 + r)^t}})(1 - r_c)
     - L_d \frac{1 - (r + 1)^{-n}}{r}
\end{equation}

Notably, in equation \ref{eq:NPV_ALL}, the same rate is used for the loan and
the cash flows, $r$. This is because the capital cost of the loan, is used to
discount the investment.

Here we see that there are several variables that affect this inequality, as
summarized in table \ref {Table:variables}.

\begin{table}
\begin{center}
\caption{Summary of variables as used in equation \ref{eq:NPV_ALL}.}
\label{Table:variables}
\begin{tabular}{ |l|p{4in}| }
 \hline
$L_d$  & the size of the down payment                                      \\
\hline
$r$    & The interest rate of the loan, which also the discount rate for the
         cash flows.                                                       \\
\hline
$n$    & The count of down payments                                        \\
\hline
$B_t$  & The benefits of the investment                                    \\
\hline
$C_t$  & The costs of the investment                                       \\
\hline
$N$    & The amount of cash flow periods                                   \\
\hline
$r_c$  & Capital gains tax.                                                \\
 \hline
\end{tabular}
\end{center}
\end{table}

In practice this inequality can be hard to approach and solve, because the
required variables can't be sufficiently estimated. For instance, the return(s)
of the investment is diffuse or that the interest rate of the loan may change
outside acceptable levels. One approach to mitigate this is a sensitivity
analysis.

\section{Other Factors}

A critical aspect is liquidity. In this paper the investment, in decision $D_I$,
has initial investment(s) and subsequent returning cash flow(s). This, combined
with other relevant cash flows, most notably the down payment of the loan, means
there can be a liquidity problem. Stated differently, some period(s) can have a
negative cash flow. One way to investigate is to net all the cash flow periods,
if they can be estimated sufficiently.

The nature of the rates need to be considered. One may be variable and the other
fixed, and hence inflation may become a factor. For instance, the loan may have
a variable rate and hence updating to inflation and monetary policy, while the
investment is fixed. The calculations need to take this into account.

But there's also psychological aspects, affected by cognitive biases. Having a
loan involves risks, most notably not being able to pay the down payments. The
size of this risk can vary with various factors. Still, mitigating this risk may
be psychologically soothing, which can be justifiably seen as a gain. However,
it is prudent to weight it against any financial benefit of deciding to do the
investment.

The investment, if carried out, may also carry risk, or volatility, which needs
to be taken into account.

Taxation needs to be considered for the investment: tax on capital gains might
render the investment irrational.

Another aspect is the conditions of the loan. For instance student loans in
Norway and Sweden have advantageous aspects that needs to be taken into account.
This is for instance the possibility to postpone payments.

\section{Conclusion}

Answering the question of whether to down pay is combination between
mathematically consider the various options, combined with qualitative
questions. There is no simple answer but needs to be approached analytically
and a qualitatively strategic judgment.

\end{document}
